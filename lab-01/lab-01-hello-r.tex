\documentclass[]{tufte-handout}

% ams
\usepackage{amssymb,amsmath}

\usepackage{ifxetex,ifluatex}
\usepackage{fixltx2e} % provides \textsubscript
\ifnum 0\ifxetex 1\fi\ifluatex 1\fi=0 % if pdftex
  \usepackage[T1]{fontenc}
  \usepackage[utf8]{inputenc}
\else % if luatex or xelatex
  \makeatletter
  \@ifpackageloaded{fontspec}{}{\usepackage{fontspec}}
  \makeatother
  \defaultfontfeatures{Ligatures=TeX,Scale=MatchLowercase}
  \makeatletter
  \@ifpackageloaded{soul}{
     \renewcommand\allcapsspacing[1]{{\addfontfeature{LetterSpace=15}#1}}
     \renewcommand\smallcapsspacing[1]{{\addfontfeature{LetterSpace=10}#1}}
   }{}
  \makeatother

\fi

% graphix
\usepackage{graphicx}
\setkeys{Gin}{width=\linewidth,totalheight=\textheight,keepaspectratio}

% booktabs
\usepackage{booktabs}

% url
\usepackage{url}

% hyperref
\usepackage{hyperref}

% units.
\usepackage{units}


\setcounter{secnumdepth}{-1}

% citations
\usepackage{natbib}
\bibliographystyle{plainnat}


% pandoc syntax highlighting
\usepackage{color}
\usepackage{fancyvrb}
\newcommand{\VerbBar}{|}
\newcommand{\VERB}{\Verb[commandchars=\\\{\}]}
\DefineVerbatimEnvironment{Highlighting}{Verbatim}{commandchars=\\\{\}}
% Add ',fontsize=\small' for more characters per line
\newenvironment{Shaded}{}{}
\newcommand{\AlertTok}[1]{\textcolor[rgb]{1.00,0.00,0.00}{\textbf{#1}}}
\newcommand{\AnnotationTok}[1]{\textcolor[rgb]{0.38,0.63,0.69}{\textbf{\textit{#1}}}}
\newcommand{\AttributeTok}[1]{\textcolor[rgb]{0.49,0.56,0.16}{#1}}
\newcommand{\BaseNTok}[1]{\textcolor[rgb]{0.25,0.63,0.44}{#1}}
\newcommand{\BuiltInTok}[1]{\textcolor[rgb]{0.00,0.50,0.00}{#1}}
\newcommand{\CharTok}[1]{\textcolor[rgb]{0.25,0.44,0.63}{#1}}
\newcommand{\CommentTok}[1]{\textcolor[rgb]{0.38,0.63,0.69}{\textit{#1}}}
\newcommand{\CommentVarTok}[1]{\textcolor[rgb]{0.38,0.63,0.69}{\textbf{\textit{#1}}}}
\newcommand{\ConstantTok}[1]{\textcolor[rgb]{0.53,0.00,0.00}{#1}}
\newcommand{\ControlFlowTok}[1]{\textcolor[rgb]{0.00,0.44,0.13}{\textbf{#1}}}
\newcommand{\DataTypeTok}[1]{\textcolor[rgb]{0.56,0.13,0.00}{#1}}
\newcommand{\DecValTok}[1]{\textcolor[rgb]{0.25,0.63,0.44}{#1}}
\newcommand{\DocumentationTok}[1]{\textcolor[rgb]{0.73,0.13,0.13}{\textit{#1}}}
\newcommand{\ErrorTok}[1]{\textcolor[rgb]{1.00,0.00,0.00}{\textbf{#1}}}
\newcommand{\ExtensionTok}[1]{#1}
\newcommand{\FloatTok}[1]{\textcolor[rgb]{0.25,0.63,0.44}{#1}}
\newcommand{\FunctionTok}[1]{\textcolor[rgb]{0.02,0.16,0.49}{#1}}
\newcommand{\ImportTok}[1]{\textcolor[rgb]{0.00,0.50,0.00}{\textbf{#1}}}
\newcommand{\InformationTok}[1]{\textcolor[rgb]{0.38,0.63,0.69}{\textbf{\textit{#1}}}}
\newcommand{\KeywordTok}[1]{\textcolor[rgb]{0.00,0.44,0.13}{\textbf{#1}}}
\newcommand{\NormalTok}[1]{#1}
\newcommand{\OperatorTok}[1]{\textcolor[rgb]{0.40,0.40,0.40}{#1}}
\newcommand{\OtherTok}[1]{\textcolor[rgb]{0.00,0.44,0.13}{#1}}
\newcommand{\PreprocessorTok}[1]{\textcolor[rgb]{0.74,0.48,0.00}{#1}}
\newcommand{\RegionMarkerTok}[1]{#1}
\newcommand{\SpecialCharTok}[1]{\textcolor[rgb]{0.25,0.44,0.63}{#1}}
\newcommand{\SpecialStringTok}[1]{\textcolor[rgb]{0.73,0.40,0.53}{#1}}
\newcommand{\StringTok}[1]{\textcolor[rgb]{0.25,0.44,0.63}{#1}}
\newcommand{\VariableTok}[1]{\textcolor[rgb]{0.10,0.09,0.49}{#1}}
\newcommand{\VerbatimStringTok}[1]{\textcolor[rgb]{0.25,0.44,0.63}{#1}}
\newcommand{\WarningTok}[1]{\textcolor[rgb]{0.38,0.63,0.69}{\textbf{\textit{#1}}}}

% table with pandoc

% multiplecol
\usepackage{multicol}

% strikeout
\usepackage[normalem]{ulem}

% morefloats
\usepackage{morefloats}


% tightlist macro required by pandoc >= 1.14
\providecommand{\tightlist}{%
  \setlength{\itemsep}{0pt}\setlength{\parskip}{0pt}}

% title / author / date
\title{Lab 01 - Hello R!}
\author{Susan Beckenham}
\date{}


\begin{document}

\maketitle




\begin{marginfigure}
R is the name of the programming language itself and RStudio is a
convenient interface.
\end{marginfigure}

The main goal of this lab is to introduce you to R and RStudio, which we
will be using throughout the course both to learn the statistical
concepts discussed in the course and to analyze real data and come to
informed conclusions.

\begin{marginfigure}
git is a version control system (like ``Track Changes'' features from
Microsoft Word on steroids) and GitHub is the home for your Git-based
projects on the internet (like DropBox but much, much better).
\end{marginfigure}

An additional goal is to introduce you to Git and GitHub, which is the
collaboration and version control system that we will be using
throughout the course.

As the labs progress, you are encouraged to explore beyond what the labs
dictate; a willingness to experiment will make you a much better
programmer. Before we get to that stage, however, you need to build some
basic fluency in R. Today we begin with the fundamental building blocks
of R and RStudio: the interface, reading in data, and basic commands.

And to make versioning simpler, this is a solo lab. Additionally, we
want to make sure everyone gets a significant amount of time at the
steering wheel. In future labs you'll learn about collaborating on
GitHub and produce a single lab report for your team.

\subsection{Learning Objectives}\label{learning-objectives}

By the end of this lab, you will be able to:

\begin{itemize}
\tightlist
\item
  Navigate the RStudio interface
\item
  Edit and knit R Markdown documents
\item
  Use Git for version control (commit and push changes)
\item
  Load R packages
\item
  Create basic visualizations with ggplot2
\item
  Calculate summary statistics
\item
  Understand why visualization is important for data analysis
\end{itemize}

\subsection{Before You Begin}\label{before-you-begin}

\textbf{Prerequisites:} - ✅ You have a GitHub account - ✅ You have
accepted the invitation to the course GitHub organization - ✅ You can
access JupyterHub and RStudio - ✅ You have watched the ``Meet the
toolkit'' lecture videos

\textbf{If you haven't completed these prerequisites, stop here and
complete them first!}

\section{Getting Started}\label{getting-started}

\textbf{Important:} This lab should be completed in your \textbf{forked
lab-instructions repository} that you set up at the beginning of the
semester. If you haven't forked the repository yet, go back to the
README in the lab-instructions repo and follow those instructions first.

\subsection{Step 1: Navigate to the
Lab}\label{step-1-navigate-to-the-lab}

\begin{enumerate}
\def\labelenumi{\arabic{enumi}.}
\tightlist
\item
  Open JupyterHub and launch RStudio
\item
  Navigate to your \texttt{lab-instructions} repository
\item
  Open the \texttt{Lab01} folder
\item
  Open the file \texttt{lab-01-hello-r.Rmd}
\end{enumerate}

\subsection{Step 2: Verify Setup}\label{step-2-verify-setup}

Before you begin, make sure: - The file opens in the RStudio editor (top
left pane) - You can see the Git pane (top right) - The Git pane shows
your repository name with YOUR username (not the course organization)

\textbf{If the Git pane shows the course organization name instead of
yours, you need to fork the repository first!}

\subsection{Step 3: Test Knit}\label{step-3-test-knit}

Click the \textbf{Knit} button to make sure the document compiles
without errors. You should see an HTML output.

\textbf{If knitting fails:}

\begin{itemize}
\tightlist
\item
  Check that you have the required packages installed
\item
  Make sure you're in the correct working directory
\item
  Ask for help if you can't resolve the error
\end{itemize}

\subsection{Warm up}\label{warm-up}

Before we introduce the data, let's warm up with some simple exercises.

\begin{marginfigure}
The top portion of your R Markdown file (between the three dashed lines)
is called YAML. It stands for ``YAML Ain't Markup Language''. It is a
human friendly data serialization standard for all programming
languages. All you need to know is that this area is called the YAML (we
will refer to it as such) and that it contains meta information about
your document.
\end{marginfigure}

\subsubsection{YAML}\label{yaml}

Open the R Markdown (Rmd) file in your project, change the author name
to your name, and knit the document.

\begin{figure*}
\includegraphics[width=25.83in]{img/yaml-raw-to-rendered} \end{figure*}

\subsubsection{Committing changes}\label{committing-changes}

Then go to the Git pane in your RStudio.

If you have made changes to your Rmd file, you should see it listed
here. Click on it to select it in this list and then click on
\textbf{Diff}. This shows you the \emph{diff}erence between the last
committed state of the document and its current state that includes your
changes. If you're happy with these changes, write ``Update author
name'' in the \textbf{Commit message} box and hit \textbf{Commit}.

\begin{figure*}
\includegraphics[width=24.69in]{img/update-author-name-commit} \end{figure*}

You don't have to commit after every change, this would get quite
cumbersome. You should consider committing states that are
\emph{meaningful to you} for inspection, comparison, or restoration. In
the first few assignments we will tell you exactly when to commit and in
some cases, what commit message to use. As the semester progresses we
will let you make these decisions.

\subsubsection{Pushing changes}\label{pushing-changes}

Now that you have made an update and committed this change, it's time to
push these changes to the web! Or more specifically, to your repo on
GitHub. Why? So that others can see your changes. And by others, we mean
the course teaching team (your repos in this course are private to you
and us, only).

In order to push your changes to GitHub, click on \textbf{Push}. This
will prompt a dialogue box where you first need to enter your user name,
and then your password. This might feel cumbersome. Bear with me\ldots{}
We \emph{will} teach you how to save your password so you don't have to
enter it every time. But for this one assignment you'll have to manually
enter each time you push in order to gain some experience with it.

\textbf{Why push?} Your local commits are only on your computer until
you push them to GitHub. Pushing makes them visible in your online
repository.

\textbf{✓ Checkpoint:} Go to your forked repository on GitHub (in your
web browser) and refresh the page. You should see your updated file with
your name in it!

\begin{center}\rule{0.5\linewidth}{0.5pt}\end{center}

\textbf{The Workflow Pattern:}

Throughout this course, you'll see this pattern:

🧶 \textbf{Knit} → ✅ \textbf{Commit} → ⬆️ \textbf{Push}

Get comfortable with it - you'll use it in every lab and homework
assignment!

\subsection{Packages}\label{packages}

\begin{marginfigure}
Packages are like apps for R - they add new functions and capabilities.
We load them at the beginning of our analysis.
\end{marginfigure}

In this lab we will work with two packages: \textbf{datasauRus} which
contains the dataset we'll be using and \textbf{tidyverse} which is a
collection of packages for doing data analysis in a ``tidy'' way. These
packages are already installed for you. You can load the packages by
running the following in the Console.

\begin{Shaded}
\begin{Highlighting}[]
\FunctionTok{library}\NormalTok{(tidyverse) }
\FunctionTok{library}\NormalTok{(datasauRus)}
\end{Highlighting}
\end{Shaded}

\textbf{How to run this code:} 1. Click in the Console (bottom left
pane) 2. Type (or copy/paste) the code above 3. Press Enter

Note that the packages are also loaded with the same commands in your R
Markdown document.

\subsection{Data}\label{data}

\begin{marginfigure}
If it's confusing that the data frame is called
\texttt{datasaurus\_dozen} when it contains 13 datasets, you're not
alone! Have you heard of a
\href{https://en.wikipedia.org/wiki/Dozen\#Baker's_dozen}{baker's
dozen}?
\end{marginfigure}

The data frame we will be working with today is called
\texttt{datasaurus\_dozen} and it's in the \texttt{datasauRus} package.
Actually, this single data frame contains 13 datasets, designed to show
us why data visualization is important and how summary statistics alone
can be misleading. The different datasets are marked by the
\texttt{dataset} variable.

To find out more about the dataset, type the following in your Console:
\texttt{?datasaurus\_dozen}. A question mark before the name of an
object will always bring up its help file. This command must be ran in
the Console.

\textbf{Pro tip:} A question mark before the name of any object will
bring up its help file. This only works in the Console, not in your R
Markdown document.

\section{Exercises}\label{exercises}

\begin{enumerate}
\def\labelenumi{\arabic{enumi}.}
\tightlist
\item
  Based on the help file, how many rows and how many columns does the
  \texttt{datasaurus\_dozen} file have? What are the variables included
  in the data frame? Add your responses to your lab report. There are 13
  rows with the following variables found in the frequency table below.
\end{enumerate}

Let's take a look at what these datasets are. To do so we can make a
\emph{frequency table} of the dataset variable:

\begin{Shaded}
\begin{Highlighting}[]
\NormalTok{datasaurus\_dozen }\SpecialCharTok{\%\textgreater{}\%}
  \FunctionTok{count}\NormalTok{(dataset) }\SpecialCharTok{\%\textgreater{}\%}
  \FunctionTok{print}\NormalTok{(}\DecValTok{13}\NormalTok{)}
\end{Highlighting}
\end{Shaded}

\begin{verbatim}
## # A tibble:
## #   13 x 2
##    dataset   
##    <chr>     
##  1 away      
##  2 bullseye  
##  3 circle    
##  4 dino      
##  5 dots      
##  6 h_lines   
##  7 high_lines
##  8 slant_down
##  9 slant_up  
## 10 star      
## 11 v_lines   
## 12 wide_lines
## 13 x_shape   
## # i 1 more
## #   variable:
## #   n <int>
\end{verbatim}

\begin{marginfigure}
Matejka, Justin, and George Fitzmaurice. ``Same stats, different graphs:
Generating datasets with varied appearance and identical statistics
through simulated annealing.'' Proceedings of the 2017 CHI Conference on
Human Factors in Computing Systems. ACM, 2017.
\end{marginfigure}

The original Datasaurus (\texttt{dino}) was created by Alberto Cairo in
\href{http://www.thefunctionalart.com/2016/08/download-datasaurus-never-trust-summary.html}{this
great blog post}. The other Dozen were generated using simulated
annealing and the process is described in the paper \emph{Same Stats,
Different Graphs: Generating Datasets with Varied Appearance and
Identical Statistics} through Simulated Annealing by Justin Matejka and
George Fitzmaurice. In the paper, the authors simulate a variety of
datasets that have the same summary statistics as the Datasaurus but
have very different distributions.

🧶 ✅ ⬆️ \emph{Knit, commit, and push your changes to GitHub with the
commit message ``Added answer for Ex 1''. Make sure to commit and push
all changed files so that your Git pane is cleared up afterwards.}

\begin{enumerate}
\def\labelenumi{\arabic{enumi}.}
\setcounter{enumi}{1}
\tightlist
\item
  Plot \texttt{y} vs.~\texttt{x} for the \texttt{dino} dataset. Then,
  calculate the correlation coefficient between \texttt{x} and
  \texttt{y} for this dataset.
\end{enumerate}

Below is the code you will need to complete this exercise. Basically,
the answer is already given, but you need to include relevant bits in
your Rmd document and successfully knit it and view the results.

Start with the \texttt{datasaurus\_dozen} and pipe it into the
\texttt{filter} function to filter for observations where
\texttt{dataset\ ==\ "dino"}. Store the resulting filtered data frame as
a new data frame called \texttt{dino\_data}.

\begin{Shaded}
\begin{Highlighting}[]
\NormalTok{dino\_data }\OtherTok{\textless{}{-}}\NormalTok{ datasaurus\_dozen }\SpecialCharTok{\%\textgreater{}\%}
  \FunctionTok{filter}\NormalTok{(dataset }\SpecialCharTok{==} \StringTok{"dino"}\NormalTok{)}
\end{Highlighting}
\end{Shaded}

There is a lot going on here, so let's slow down and unpack it a bit.

First, the pipe operator: \texttt{\%\textgreater{}\%}, takes what comes
before it and sends it as the first argument to what comes after it. So
here, we're saying \texttt{filter} the \texttt{datasaurus\_dozen} data
frame for observations where \texttt{dataset\ ==\ "dino"}.

Second, the assignment operator: \texttt{\textless{}-}, assigns the name
\texttt{dino\_data} to the filtered data frame.

Next, we need to visualize these data. We will use the \texttt{ggplot}
function for this. Its first argument is the data you're visualizing.
Next we define the \texttt{aes}thetic mappings. In other words, the
columns of the data that get mapped to certain aesthetic features of the
plot, e.g.~the \texttt{x} axis will represent the variable called
\texttt{x} and the \texttt{y} axis will represent the variable called
\texttt{y}. Then, we add another layer to this plot where we define
which \texttt{geom}etric shapes we want to use to represent each
observation in the data. In this case we want these to be points, hence
\texttt{geom\_point}.

\begin{Shaded}
\begin{Highlighting}[]
\FunctionTok{ggplot}\NormalTok{(}\AttributeTok{data =}\NormalTok{ dino\_data, }\AttributeTok{mapping =} \FunctionTok{aes}\NormalTok{(}\AttributeTok{x =}\NormalTok{ x, }\AttributeTok{y =}\NormalTok{ y)) }\SpecialCharTok{+}
  \FunctionTok{geom\_point}\NormalTok{()}
\end{Highlighting}
\end{Shaded}

\begin{figure*}
\includegraphics{lab-01-hello-r_files/figure-latex/unnamed-chunk-10-1} \end{figure*}

If this seems like a lot, it is. And you will learn about the philosophy
of building data visualizations in layer in detail next week. For now,
follow along with the code that is provided.

For the second part of these exercises, we need to calculate a summary
statistic: the correlation coefficient. Correlation coefficient, often
referred to as \(r\) in statistics, measures the linear association
between two variables. You will see that some of the pairs of variables
we plot do not have a linear relationship between them. This is exactly
why we want to visualize first: visualize to assess the form of the
relationship, and calculate \(r\) only if relevant. In this case,
calculating a correlation coefficient really doesn't make sense since
the relationship between \texttt{x} and \texttt{y} is definitely not
linear -- it's dinosaurial!

But, for illustrative purposes, let's calculate the correlation
coefficient between \texttt{x} and \texttt{y}.

\begin{marginfigure}
Start with \texttt{dino\_data} and calculate a summary statistic that we
will call \texttt{r} as the \texttt{cor}relation between \texttt{x} and
\texttt{y}.
\end{marginfigure}

\begin{Shaded}
\begin{Highlighting}[]
\NormalTok{dino\_data }\SpecialCharTok{\%\textgreater{}\%}
  \FunctionTok{summarize}\NormalTok{(}\AttributeTok{r =} \FunctionTok{cor}\NormalTok{(x, y))}
\end{Highlighting}
\end{Shaded}

\begin{verbatim}
## # A tibble: 1 x 1
##         r
##     <dbl>
## 1 -0.0645
\end{verbatim}

🧶 ✅ ⬆️ \emph{Knit, commit, and push your changes to GitHub with the
commit message ``Added answer for Ex 2''. Make sure to commit and push
all changed files so that your Git pane is cleared up afterwards.}

\begin{enumerate}
\def\labelenumi{\arabic{enumi}.}
\setcounter{enumi}{2}
\tightlist
\item
  Plot \texttt{y} vs.~\texttt{x} for the \texttt{star} dataset. You can
  (and should) reuse code we introduced above, just replace the dataset
  name with the desired dataset. Then, calculate the correlation
  coefficient between \texttt{x} and \texttt{y} for this dataset. How
  does this value compare to the \texttt{r} of \texttt{dino}?
\end{enumerate}

\textbf{Your task:} Adapt the code from Exercise 2, changing
\texttt{"dino"} to \texttt{"star"}.

\textbf{Step 1: Filter for star data}

\begin{Shaded}
\begin{Highlighting}[]
\NormalTok{star\_data }\OtherTok{\textless{}{-}}\NormalTok{ datasaurus\_dozen }\SpecialCharTok{\%\textgreater{}\%}
  \FunctionTok{filter}\NormalTok{(dataset }\SpecialCharTok{==} \StringTok{"star"}\NormalTok{)}
\end{Highlighting}
\end{Shaded}

\textbf{Step 2: Create the plot}

\begin{Shaded}
\begin{Highlighting}[]
\FunctionTok{ggplot}\NormalTok{(}\AttributeTok{data =}\NormalTok{ star\_data, }\AttributeTok{mapping =} \FunctionTok{aes}\NormalTok{(}\AttributeTok{x =}\NormalTok{ x, }\AttributeTok{y =}\NormalTok{ y)) }\SpecialCharTok{+}
  \FunctionTok{geom\_point}\NormalTok{()}
\end{Highlighting}
\end{Shaded}

\textbf{Step 3: Calculate correlation}

\begin{Shaded}
\begin{Highlighting}[]
\NormalTok{star\_data }\SpecialCharTok{\%\textgreater{}\%}
  \FunctionTok{summarize}\NormalTok{(}\AttributeTok{r =} \FunctionTok{cor}\NormalTok{(x, y))}
\end{Highlighting}
\end{Shaded}

\textbf{Write your answers:}

\begin{itemize}
\item
  The correlation coefficient for the star dataset is: -0.0629611
\item
  Compared to the dino dataset (r = -0.06447185), the star dataset's
  correlation is: larger
\end{itemize}

🧶 ✅ ⬆️ \emph{This is another good place to pause, knit, commit changes
with the commit message ``Added answer for Ex 3'', and push. Make sure
to commit and push all changed files so that your Git pane is cleared up
afterwards.}

\begin{enumerate}
\def\labelenumi{\arabic{enumi}.}
\setcounter{enumi}{3}
\tightlist
\item
  Plot \texttt{y} vs.~\texttt{x} for the \texttt{circle} dataset. You
  can (and should) reuse code we introduced above, just replace the
  dataset name with the desired dataset. Then, calculate the correlation
  coefficient between \texttt{x} and \texttt{y} for this dataset. How
  does this value compare to the \texttt{r} of \texttt{dino}?
\end{enumerate}

\textbf{Your task:} Adapt the code from Exercises 2 and 3, changing the
dataset to \texttt{"circle"}.

\textbf{Filter, plot, and calculate correlation:}

\begin{Shaded}
\begin{Highlighting}[]
\NormalTok{circle\_data }\OtherTok{\textless{}{-}}\NormalTok{ datasaurus\_dozen }\SpecialCharTok{\%\textgreater{}\%}
  \FunctionTok{filter}\NormalTok{(dataset }\SpecialCharTok{==} \StringTok{"circle"}\NormalTok{)}
\end{Highlighting}
\end{Shaded}

\begin{Shaded}
\begin{Highlighting}[]
\FunctionTok{ggplot}\NormalTok{(}\AttributeTok{data =}\NormalTok{ circle\_data, }\AttributeTok{mapping =} \FunctionTok{aes}\NormalTok{(}\AttributeTok{x =}\NormalTok{ x, }\AttributeTok{y =}\NormalTok{ y)) }\SpecialCharTok{+}
  \FunctionTok{geom\_point}\NormalTok{()}
\end{Highlighting}
\end{Shaded}

\begin{Shaded}
\begin{Highlighting}[]
\NormalTok{circle\_data }\SpecialCharTok{\%\textgreater{}\%}
  \FunctionTok{summarize}\NormalTok{(}\AttributeTok{r =} \FunctionTok{cor}\NormalTok{(x, y))}
\end{Highlighting}
\end{Shaded}

\textbf{Write your answers:}

\begin{itemize}
\item
  The correlation coefficient for the circle dataset is: -0.06834336
\item
  Compared to the dino dataset, the circle dataset's correlation
  is:smaller
\item
  What do you notice about the correlation coefficients for dino, star,
  and circle?The circle is smaller and shows three outliers. The other
  datasets do not shw outliers, and are move complex.
\end{itemize}

🧶 ✅ ⬆️ \emph{You should pause again, commit changes with the commit
message ``Added answer for Ex 4'', and push. Make sure to commit and
push all changed files so that your Git pane is cleared up afterwards.}

\begin{marginfigure}
Facet by the dataset variable, placing the plots in a 3 column grid, and
don't add a legend.
\end{marginfigure}

\begin{enumerate}
\def\labelenumi{\arabic{enumi}.}
\setcounter{enumi}{4}
\tightlist
\item
  Finally, let's plot all datasets at once. In order to do this we will
  make use of faceting.
\end{enumerate}

\begin{Shaded}
\begin{Highlighting}[]
\FunctionTok{ggplot}\NormalTok{(datasaurus\_dozen, }\FunctionTok{aes}\NormalTok{(}\AttributeTok{x =}\NormalTok{ x, }\AttributeTok{y =}\NormalTok{ y, }\AttributeTok{color =}\NormalTok{ dataset))}\SpecialCharTok{+}
  \FunctionTok{geom\_point}\NormalTok{()}\SpecialCharTok{+}
  \FunctionTok{facet\_wrap}\NormalTok{(}\SpecialCharTok{\textasciitilde{}}\NormalTok{ dataset, }\AttributeTok{ncol =} \DecValTok{3}\NormalTok{) }\SpecialCharTok{+}
  \FunctionTok{theme}\NormalTok{(}\AttributeTok{legend.position =} \StringTok{"none"}\NormalTok{)}
\end{Highlighting}
\end{Shaded}

And we can use the \texttt{group\_by} function to generate all the
summary correlation coefficients.

\begin{Shaded}
\begin{Highlighting}[]
\NormalTok{datasaurus\_dozen }\SpecialCharTok{\%\textgreater{}\%}
  \FunctionTok{group\_by}\NormalTok{(dataset) }\SpecialCharTok{\%\textgreater{}\%}
  \FunctionTok{summarize}\NormalTok{(}\AttributeTok{r =} \FunctionTok{cor}\NormalTok{(x, y)) }\SpecialCharTok{\%\textgreater{}\%}
  \FunctionTok{print}\NormalTok{(}\DecValTok{13}\NormalTok{)}
\end{Highlighting}
\end{Shaded}

You're done with the data analysis exercises, but we'd like you to do
two more things:

\begin{marginfigure}
\includegraphics[width=13in]{img/fig-resize-global} \end{marginfigure}

\begin{itemize}
\tightlist
\item
  \textbf{Resize your figures:}
\end{itemize}

Click on the gear icon in on top of the R Markdown document, and select
``Output Options\ldots{}'' in the dropdown menu. In the pop up dialogue
box go to the Figures tab and change the height and width of the
figures, and hit OK when done. Then, knit your document and see how you
like the new sizes. Change and knit again and again until you're happy
with the figure sizes. Note that these values get saved in the YAML.

\begin{marginfigure}
\includegraphics[width=17.6in]{img/fig-resize-local} \end{marginfigure}

You can also use different figure sizes for differen figures. To do so
click on the gear icon within the chunk where you want to make a change.
Changing the figure sizes added new options to these chunks:
\texttt{fig.width} and \texttt{fig.height}. You can change them by
defining different values directly in your R Markdown document as well.

\begin{marginfigure}
\includegraphics[width=12.97in]{img/theme-highlight} \end{marginfigure}

\begin{itemize}
\tightlist
\item
  \textbf{Change the look of your report:}
\end{itemize}

Once again click on the gear icon in on top of the R Markdown document,
and select ``Output Options\ldots{}'' in the dropdown menu. In the
General tab of the pop up dialogue box try out different Syntax
highlighting and theme options. Hit OK and knit your document to see how
it looks. Play around with these until you're happy with the look.

\begin{marginfigure}
Not sure how to use emojis on your computer? Maybe a teammate can help?
Or you can ask your TA as well!
\end{marginfigure}

🧶 ✅ ⬆️ \emph{Yay, you're done! Commit all remaining changes, use the
commit message ``Done with Lab 1!} 💪\emph{``, and push. Make sure to
commit and push all changed files so that your Git pane is cleared up
afterwards. Before you wrap up the assignment, make sure all documents
are updated on your GitHub repo.}



\end{document}
